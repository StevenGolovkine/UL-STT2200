% Options for packages loaded elsewhere
% Options for packages loaded elsewhere
\PassOptionsToPackage{unicode}{hyperref}
\PassOptionsToPackage{hyphens}{url}
\PassOptionsToPackage{dvipsnames,svgnames,x11names}{xcolor}
%
\documentclass[
  french,
  letterpaper,
  DIV=11,
  numbers=noendperiod]{scrartcl}
\usepackage{xcolor}
\usepackage{amsmath,amssymb}
\setcounter{secnumdepth}{-\maxdimen} % remove section numbering
\usepackage{iftex}
\ifPDFTeX
  \usepackage[T1]{fontenc}
  \usepackage[utf8]{inputenc}
  \usepackage{textcomp} % provide euro and other symbols
\else % if luatex or xetex
  \usepackage{unicode-math} % this also loads fontspec
  \defaultfontfeatures{Scale=MatchLowercase}
  \defaultfontfeatures[\rmfamily]{Ligatures=TeX,Scale=1}
\fi
\usepackage{lmodern}
\ifPDFTeX\else
  % xetex/luatex font selection
\fi
% Use upquote if available, for straight quotes in verbatim environments
\IfFileExists{upquote.sty}{\usepackage{upquote}}{}
\IfFileExists{microtype.sty}{% use microtype if available
  \usepackage[]{microtype}
  \UseMicrotypeSet[protrusion]{basicmath} % disable protrusion for tt fonts
}{}
\makeatletter
\@ifundefined{KOMAClassName}{% if non-KOMA class
  \IfFileExists{parskip.sty}{%
    \usepackage{parskip}
  }{% else
    \setlength{\parindent}{0pt}
    \setlength{\parskip}{6pt plus 2pt minus 1pt}}
}{% if KOMA class
  \KOMAoptions{parskip=half}}
\makeatother
% Make \paragraph and \subparagraph free-standing
\makeatletter
\ifx\paragraph\undefined\else
  \let\oldparagraph\paragraph
  \renewcommand{\paragraph}{
    \@ifstar
      \xxxParagraphStar
      \xxxParagraphNoStar
  }
  \newcommand{\xxxParagraphStar}[1]{\oldparagraph*{#1}\mbox{}}
  \newcommand{\xxxParagraphNoStar}[1]{\oldparagraph{#1}\mbox{}}
\fi
\ifx\subparagraph\undefined\else
  \let\oldsubparagraph\subparagraph
  \renewcommand{\subparagraph}{
    \@ifstar
      \xxxSubParagraphStar
      \xxxSubParagraphNoStar
  }
  \newcommand{\xxxSubParagraphStar}[1]{\oldsubparagraph*{#1}\mbox{}}
  \newcommand{\xxxSubParagraphNoStar}[1]{\oldsubparagraph{#1}\mbox{}}
\fi
\makeatother


\usepackage{longtable,booktabs,array}
\usepackage{calc} % for calculating minipage widths
% Correct order of tables after \paragraph or \subparagraph
\usepackage{etoolbox}
\makeatletter
\patchcmd\longtable{\par}{\if@noskipsec\mbox{}\fi\par}{}{}
\makeatother
% Allow footnotes in longtable head/foot
\IfFileExists{footnotehyper.sty}{\usepackage{footnotehyper}}{\usepackage{footnote}}
\makesavenoteenv{longtable}
\usepackage{graphicx}
\makeatletter
\newsavebox\pandoc@box
\newcommand*\pandocbounded[1]{% scales image to fit in text height/width
  \sbox\pandoc@box{#1}%
  \Gscale@div\@tempa{\textheight}{\dimexpr\ht\pandoc@box+\dp\pandoc@box\relax}%
  \Gscale@div\@tempb{\linewidth}{\wd\pandoc@box}%
  \ifdim\@tempb\p@<\@tempa\p@\let\@tempa\@tempb\fi% select the smaller of both
  \ifdim\@tempa\p@<\p@\scalebox{\@tempa}{\usebox\pandoc@box}%
  \else\usebox{\pandoc@box}%
  \fi%
}
% Set default figure placement to htbp
\def\fps@figure{htbp}
\makeatother



\ifLuaTeX
\usepackage[bidi=basic]{babel}
\else
\usepackage[bidi=default]{babel}
\fi
% get rid of language-specific shorthands (see #6817):
\let\LanguageShortHands\languageshorthands
\def\languageshorthands#1{}


\setlength{\emergencystretch}{3em} % prevent overfull lines

\providecommand{\tightlist}{%
  \setlength{\itemsep}{0pt}\setlength{\parskip}{0pt}}



 


\KOMAoption{captions}{tableheading}
\makeatletter
\@ifpackageloaded{caption}{}{\usepackage{caption}}
\AtBeginDocument{%
\ifdefined\contentsname
  \renewcommand*\contentsname{Table des matières}
\else
  \newcommand\contentsname{Table des matières}
\fi
\ifdefined\listfigurename
  \renewcommand*\listfigurename{Liste des Figures}
\else
  \newcommand\listfigurename{Liste des Figures}
\fi
\ifdefined\listtablename
  \renewcommand*\listtablename{Liste des Tables}
\else
  \newcommand\listtablename{Liste des Tables}
\fi
\ifdefined\figurename
  \renewcommand*\figurename{Figure}
\else
  \newcommand\figurename{Figure}
\fi
\ifdefined\tablename
  \renewcommand*\tablename{Table}
\else
  \newcommand\tablename{Table}
\fi
}
\@ifpackageloaded{float}{}{\usepackage{float}}
\floatstyle{ruled}
\@ifundefined{c@chapter}{\newfloat{codelisting}{h}{lop}}{\newfloat{codelisting}{h}{lop}[chapter]}
\floatname{codelisting}{Listing}
\newcommand*\listoflistings{\listof{codelisting}{Liste des Listings}}
\makeatother
\makeatletter
\makeatother
\makeatletter
\@ifpackageloaded{caption}{}{\usepackage{caption}}
\@ifpackageloaded{subcaption}{}{\usepackage{subcaption}}
\makeatother
\usepackage{bookmark}
\IfFileExists{xurl.sty}{\usepackage{xurl}}{} % add URL line breaks if available
\urlstyle{same}
\hypersetup{
  pdftitle={Évaluations},
  pdflang={fr},
  colorlinks=true,
  linkcolor={blue},
  filecolor={Maroon},
  citecolor={Blue},
  urlcolor={Blue},
  pdfcreator={LaTeX via pandoc}}


\title{Évaluations}
\author{}
\date{}
\begin{document}
\maketitle


\subsection{Modalités d'évaluations}\label{modalituxe9s-duxe9valuations}

\begin{longtable}[]{@{}llll@{}}
\toprule\noalign{}
Titre & Date & Mode de travail & Pondération \\
\midrule\noalign{}
\endhead
\bottomrule\noalign{}
\endlastfoot
Examen 1 & & Individuel & 25\% \\
Examen 2 & & Individuel & 50\% \\
Projet & & En équipe & 25\% \\
\end{longtable}

Identification. Lors d'un examen, une carte d'identité avec photo
admissible doit être déposée sur le coin de votre table. Les cartes
admissibles sont la carte de l'Université Laval en plastique, un permis
de conduire canadien, une carte d'assurance-maladie avec photo émise par
une province canadienne ou un passeport canadien ou étranger.

\subsection{Informations détaillées sur les
évaluations}\label{informations-duxe9tailluxe9es-sur-les-uxe9valuations}

\subsubsection{Examen 1}\label{examen-1}

Date et lieu:

Mode de travail: Individuel

Pondération: 25\%

Remise de l'évaluation:

Directives de l'évaluation: Tout le contenu du cours est susceptible
d'être a l'examen.

Matériel autorisé: pas de restriction.

\subsubsection{Examen 2}\label{examen-2}

Date et lieu:

Mode de travail: Individuel

Pondération: 50\%

Remise de l'évaluation:

Directives de l'évaluation: Tout le contenu du cours est susceptible
d'être à l'examen.

Matériel autorisé: pas de restriction.

\subsubsection{Projet}\label{projet}

Date et lieu:

Mode de travail: En équipe

Pondération: 25\%

Remise de l'évaluation:

Directives de l'évaluation:

Matériel autorisé: pas de restriction.

\subsection{Échelle des cotes}\label{uxe9chelle-des-cotes}

\begin{longtable}[]{@{}crr@{}}
\toprule\noalign{}
Cote & \% minimum & \% maximum \\
\midrule\noalign{}
\endhead
\bottomrule\noalign{}
\endlastfoot
A+ & 92 & 100 \\
A & 88 & 91.99 \\
A- & 84 & 87.99 \\
B+ & 80 & 83.99 \\
B & 75 & 79.99 \\
B- & 70 & 74.99 \\
C+ & 65 & 69.99 \\
C & 60 & 64.99 \\
C- & 55 & 59.99 \\
D+ & 52 & 54.99 \\
D & 50 & 51.99 \\
E & 0 & 49.99 \\
\end{longtable}

\subsection{Détails sur les modalités
d'évaluation}\label{duxe9tails-sur-les-modalituxe9s-duxe9valuation}

Conformément à la politique du Département de mathématiques et de
statistique en matière d'amélioration et de consolidation de la
connaissance du français, la qualité de l'écrit sera sanctionnée dans
tous les travaux et examens. Un maximum de 10\% des points pourra être
enlevé pour la qualité de la langue et de la rédaction.

Aucun retard n'est accepté pour la remise des travaux.

Toute reprise d'évaluation accordée par la politique de reprise
d'évaluation du Département aura lieu lors des dates de reprises
officielles de la faculté.

\subsection{Absence à une activité
obligatoire}\label{absence-uxe0-une-activituxe9-obligatoire}

Ces modalités s'appliquent en vertu des articles 4.41 et 4.42 du
Réglement des études de l'Université Laval:

\begin{itemize}
\tightlist
\item
  4.41 Tout défaut de se soumettre à une activité d' évaluation entraine
  la note zéro pour cette activité d' évaluation, à moins que l'
  étudiante ou l' étudiant ne démontre que cette omission est
  attribuable à des motifs sérieux.
\item
  4.42 La reprise d' une évaluation est possible pour des motifs
  sérieux. Elle se fait selon les modalités prévues par l' unité
  responsable de l' activité de formation.
\end{itemize}

La reprise d' une évaluation peut donc execptionnellement être autorisée
pour des motifs jugés sérieux, dans la mesure où la procédure décrite
ci-dessous est respectée.

\textbf{Motifs d'absence jugés sérieux}

Les motifs suivants sont jugés sérieux et donc acceptables pour demander
une reprise d'évaluation:

\begin{itemize}
\tightlist
\item
  Maladie ou accident empêchant de se déplacer;
\item
  Hospitalisation;
\item
  Maladie grave ou décès d'un proche;
\item
  Participation à une activité sportive de haut niveau;
\item
  Convocation en cour de justice.
\end{itemize}

\textbf{Procédure à suivre}

Dès que possible et au plus tard cinq (5) jours ouvrables après la date
de l'évaluation (ou dans certains cas, avant la date de l'évaluation,
dès que le motif sera connu), l'étudiante ou l'étudiant qui veut faire
une demande de reprise d'évaluation doit remplir et soumettre le
formulaire électronique
``\href{https://ulavaldti.sharepoint.com/sites/FSG-reprise-dexamens-et-reconnaissance-de-cours/Lists/Demande\%20de\%20reprise\%20dexamens/NewForm.aspx?Source=https\%3A\%2F\%2Fulavaldti.sharepoint.com\%2Fsites\%2Ffsg-demande-de-reprise-examen\%2FLists\%2FDemande\%2520de\%2520reprise\%2520dexamens\%2FEtudiant.aspx\%3FCT\%3D1724436857013\%26OR\%3DOWA\%252DNT\%252DMail\%26CID\%3D42d5014c\%252D8fb5\%252D8232\%252Dbd8f\%252D69c67e3a9cdd&ContentTypeId=0x010020B04CCB63D0044D860D6BB7B9C92629007488899A2536D44DA71B9101B9C72D92&RootFolder=}{Demande
de reprise d'une évaluation}'' en prenant soin d'y joindre les pièces
justificatives requises.

Pour avoir plus de détail sur les procédures à suivre et les motifs
sérieux pouvant donner droit à une reprise d'évaluation, consulter le
document
``\href{https://www.fsg.ulaval.ca/fileadmin/site_facultaire/Espace_facultaire/Étudiants/Politiques_facultaires/Modalités_reprise_évaluation_FSG_A24.pdf}{Modalités
et procédure de reprise d'une évaluation sommative à la Faculté des
sciences et de génie}'' disponible sur le site web de la FSG.

Dans certains cas, la ou le responsable du cours pourrait adopter une
procédure simplifiée de gestion des demandes de reprises d'évaluation,
tout en respectant les critères décrits dans cette politique. Dans ces
cas, des explications particulières seront données à cet effet dans le
plan de cours et présentées lors de la première séance.

\subsection{Politique sur l'utilisation d'appareils
électroniques}\label{politique-sur-lutilisation-dappareils-uxe9lectroniques}

La politique sur l' utilisation d'appareils électroniques de la Faculté
des Sciences et de Génie peut être consultée à l' adresse:
\href{https://www.fsg.ulaval.ca/fileadmin/site_facultaire/Espace_facultaire/Étudiants/Politiques_facultaires/_Calculatrices-autorisées-FSG_20250221.pdf}{lien}.

\subsection{POlitique sur le plagiat et la fraude
académique}\label{politique-sur-le-plagiat-et-la-fraude-acaduxe9mique}

\subsection{Étudiants ayant une situation de handicap liée à une
limitation
fonctionnelle}\label{uxe9tudiants-ayant-une-situation-de-handicap-liuxe9e-uxe0-une-limitation-fonctionnelle}




\end{document}
